\documentclass[12pt]{article}
\usepackage{lingmacros}
\usepackage{tree-dvips}
\begin{document}

\section*{BEMS}

La gestione dell'energia in edifici è diventata un aspetto cruciale nella società moderna. 
L'aumento dei costi dell'energia, il cambiamento climatico e la crescente sensibilità dell'opinione 
pubblica alla sostenibilità ambientale hanno portato alla necessità di sviluppare soluzioni innovative 
per ridurre i consumi energetici degli edifici. In questo contesto, i Building Energy Management Systems 
(BEMS) rappresentano una tecnologia avanzata che consente di gestire i consumi energetici degli 
edifici in modo intelligente e automatizzato. 

L'obiettivo di questa tesi è di sviluppare un controllo
per i BEMS basato su tecniche di Reinforcement Learning (RL). In particolare, tale controllo sarà applicato 
a una unità trattamento aria (UTA) appartenente allo stabilimento Stellantis di Melfi (PZ). 
Tale stabilimento, precedentemente noto come SATA (società Automobilistica Tecnologie Avanzate Sp.A.),
è un sito produttivo e un complesso industriale del gruppo FCA Italy, controllato dalla multinazionale Stellantis.

Il percorso di tesi è svolto  in collaborazione con l'azienda Simetria, che ha fornito i dati dell'UTA.

La tesi è suddivisa in X capitoli. Il capitolo 1 fornisce una panoramica 






\end{document}